\anonsection{Реферат}
    Данная работа содержит \pageref{LastPage} страницу, в работе использовано 6 источников.

    В главе 1 рассматриваются основные понятия коммутативной алгебры, связанные с коммутативными
    кольцами и идеалами. Формулируются основные теоремы, связанные с ними, а также приводятся 
    решения некоторых задач из главы 1 книги~\cite{A-M}. 

    В главе 2 рассматривается понятие $A$-модуля 
    над коммутативным ассоциативным кольцом $A$ с единицей, формулируются элементарные теоремы, связанные с понятием $A$-модуля, далее
    рассматривается понятие точной последовательности модулей, тензорного и периодического 
    произведений двух модулей и их свойств. Производится вычисление тензорного произведения 
    и периодического произведения двух модулей  с помощью свободной резольвенты $A$-модуля. 
    
    Глава 3 посвящена вычислению подмодуля кручения в тензорных произведениях вида 
    $\left(\bigoplus_{s \ge 0} I^s \right) \ox_A J$, где $I, J \subset A$ --- идеалы и 
    $\left(\bigoplus_{s \ge 0} I^s \right) \ox_A M$, как $\hat A$-модуля где $M$ --- $A$-модуль.
    а также вычисление делителей нуля алгебры 
    $\bigoplus_{s \geq 0}{(I[t] + (t))^s / (t^{s + 1})}$ с покомпонетным умножением.

    Ключевые слова: идеал,
    коммутативное кольцо,
    кручение,
    модуль,
    периодическое произведение,
    тензорное произведение.
    
