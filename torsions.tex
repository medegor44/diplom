\section{Кручения в некоторых тензорных произведениях модулей}
    В задачах алгебраической геометрии, связанных с разрешением особенностей когерентных 
    алгебраических пучков, бывает необходимо 
    исследовать поведение когерентного алгебраического пучка при преобразованиях базисного 
    многообразия или схемы.
    Преобразование базисного многообразия подбирается так, чтобы трансформировать не локально 
    свободный когерентный пучок в локально
    свободный пучок на новом многообразии или схеме. 

    Локальным аналогом этой задачи является исследование свойств тензорного произведения модуля $M$ 
    над коммутативным кольцом $A$ на $A$-алгебру $\widetilde{A}$.

    В [3] автором изложена одна из возможных конструкций разрешения особенностей когерентного пучка, 
    локально сводящаяся к преобразованию $M \mapsto \widetilde{A} \otimes_A M$.
    Алгебра $\widetilde{A}$ получается при этом следующим образом: 
    $\widetilde{A} = \bigoplus_{s \geq 0} (I[t] + (t))^s / (t^{s + 1})$, где $I \subset A$ -- 
    ненулевой собственный
    идеал, $t$ -- элемент, трансцендентный над кольцом $A$;

    Рассмотрим коммутативное ассоциативное нетерово целостное кольцо $A$ с единицей. 
    Пусть $I \subset A$ идеал. 
    Алгеброй раздутия идеала $I$ назовем выражение $\widehat{A} := \bigoplus_{s \geq 0} I^s$.

    \begin{Def}
        Пусть $M$ -- произвольный $A$-модуль. Подмодулем кручения $\tors{M}$ называется множество
        \begin{equation*}
            \tors{M} = \left\{x \in M | \exists a \in A\setminus \{0\}, ax = 0 \right\}.
        \end{equation*}
    \end{Def}
    \begin{Def}
        Будем говорить, что $A$-модуль $M$ является модулем без кручения, если $\tors{M} = 0$.
    \end{Def}

    Пусть $M$ -- $A$-модуль без кручения. Поскольку тензорное произведение не является точным слева, 
    при тензорном умножении $M$ на алгебру раздутия $\widehat{A}$ в модуле $\widehat{A} \ox_A M$ 
    может возникнуть кручение. 

    Решается следующая частная задача: описать подмодуль кручения $\tors{\widehat{A} \ox_A I}$ 
    $A$-модуля $\widehat{A} \ox_A I$.

    Пусть, для простоты, идеал $I = (x, y)$ порожден элементами $x, y \in A$. Выясним как устроены
    его степени. 

    \begin{Theorem}
        Пусть $s \geq 1$, тогда $I^s = (x^s, x^{s-1}y, \dots, xy^{s-1}, y^s)$.
    \end{Theorem}
    \begin{Proof}
        Методом математической индукции. Пусть $s = 1$. Тогда $I^1 = (x, y)$ -- верно. Пусть утверждение
        верно значений $s \leq r$. При $s = r + 1$ имеем:
        \begin{equation*}
            I^{r + 1} = I^{r}I = \left\{\left(\sum_{n = 0}^{r} a_nx^ny^{n - r}\right)(b_1x + b_0y) 
            | a_n, b_m \in A\right\}.
        \end{equation*}
        Теперь, раскрывая скобки, получим
        \begin{equation*}
            I^{r + 1} = \left\{ b_0a_0y^{r + 1} + (b_1a_0 + b_0a_1)xy^{r} + \dots + 
                (b_1a_{r - 1} + b_0a_{r})x^ry + b_1a_rx^{r + 1} | a_n, b_m \in A \right\}
        \end{equation*}
        Таким образом,
        \begin{equation*}
            I^{r + 1} = (x^{r + 1}, x^ry, \dots, xy^r, y^{r + 1}),
        \end{equation*}
        Что завершает доказательство теоремы.
    \end{Proof}

    Так как тензорное произведение дистрибутивно относительно прямой суммы 
    \textbf{[сослаться на соответствующую главу]}, то справедлива цепочка равенств:

    \begin{equation*}
        \widehat{A} \ox_A I = \left(\bigoplus_{s \geq 0}{I^s}\right) \ox_A I = 
            \bigoplus_{s \geq 0}{\left(I^s \ox_A I\right)}.
    \end{equation*}

    \begin{Proposal} \label{torsIdentity}
        Пусть $\{M_j | j \in J\}$ -- семейство $A$-модулей. Кольцо $A$ -- целостное. Тогда
        \begin{equation*}
            \tors{\bigoplus_{j \in J} M_j} = \bigoplus_{j \in J} \tors{M_j}.
        \end{equation*}
    \end{Proposal}
    \begin{Proof}
        Покажем, что $\tors{\bigoplus_{j \in J} M_j} \subset \bigoplus_{j \in J} \tors{M_j}$. 
        Пусть $t \in \tors{\bigoplus_{j \in J} M_j}$. По определению, существует такое 
        $a \in A \setminus \{0\}$, что $at = 0$. Заметим, что $t = (t_0, t_1, \dots, t_j, \dots)$, где 
        только конечное число $t_j$ отлично от нуля. Так как в прямой сумме умножение производится
        покоординатно, то 
        \begin{equation*}
            at = (at_0, at_1, \dots, at_j, \dots) = 0,
        \end{equation*}
        из чего следует, что 
        \begin{equation*}
            at_1 = at_0 = \dots = at_j = \dots = 0
        \end{equation*}
        и $t_0 \in \tors{M_0}$, $t_1 \in \tors{M_1}$, \dots, $t_j \in \tors{M_j}$, \dots . 
        Таким образом, $t \in \bigoplus_{j \in J} \tors{M_j}$. Теперь покажем обратное включение.
        Пусть $t \in \bigoplus_{j \in J} \tors{M_j}$. Пусть $t_{i_1}, t_{i_2}, \dots, t_{i_k}$ 
        -- все элементы $t$, отличные от нуля. Как отмечалось ранее, их будет конечное число. По определению, 
        найдутся  $a_{i_1}, a_{i_2}, \dots, a_{i_k}$ все отличные от нуля и такие, 
        что $a_{i_1}t_{i_1} = a_{i_2}t_{i_2} = \dots = a_{i_k}t_{i_k} = 0$. Обозначим 
        $a := a_{i_1}a_{i_2}\dots a_{i_k}$. Так как кольцо целостное, то ни при каких $a_{i_l}$ их
        произведение не будет равно нулю. Тогда
        \begin{equation*}
            at_{i_l} = (a_{i_1}\dots a_{i_{l-1}}a_{i_{l + 1}}\dots a_{i_k})a_{i_l}t_{i_l} = 0,
        \end{equation*}
        что справедливо для всех $l = \bar{1,k}$. Тем самым мы показали, что $at = 0$. Значит 
        $t \in \tors{\bigoplus_{j \in J} M_j}$.
    \end{Proof}

    Теперь, воспольовавшись предложением \ref{torsIdentity}, можно записать седующее:
    \begin{equation*}
        \tors{\bigoplus_{s \geq 0} \left( I^s \ox_A I \right)} = 
        \bigoplus_{s \geq 0}\tors{ I^s \ox_A I }.
    \end{equation*}
    Таким образом, исходная задача свелась к вычислению подмодуля кручения $\tors{ I^s \ox_A I }$ 
    $A$-модуля $I^s \ox_A I$.
    \begin{Theorem}
        Пусть образующие иделала $I = (x, y)$ алгебраически независимы. 
        Тогда $\tors{I^s \ox_A I}$ описывается следующим образом:
        \begin{equation*}
            \tors{I^s \ox_A I} = \left< x^ny^{s - n} \ox x - x^{n + 1}y^{s - n - 1} \ox y |
             n = \bar{1, s-1}\right>_A
        \end{equation*}
    \end{Theorem}
    \begin{Proof}
        Так как идеалы $I^s$ и $I$ являются конечнопорожденными $A$-модулями, то, воспользовавшись свойством 
        тензорного произведения \textbf{[Ссылка на это свойство]}, имеем
        \begin{equation*}
            I^s \ox_A I = \left< x^ny^{s - n} \ox x, x^ny^{s - n} \ox y | n = \bar{0, s} \right>_A.
        \end{equation*}
        
        Пусть $\mu : I^s \ox_A I \rightarrow I^{s + 1}$ -- гомоморфизм, который действует на образующих
        следующим образом: $x^ny^{s - n} \ox x \mapsto x^{n + 1}y^{s - n}$, 
        $x^ny^{s - n} \ox y \mapsto x^ny^{s - n + 1}$.
        Тогда, $\ker \mu = \tors{I^s \ox_A I}$. Очевидно, что этот гомоморфизм сюръективен. Тогда,
        согласно теореме о гомоморфизме, $I^{s + 1} \simeq (I^s \ox_A I) / \ker \mu$. Так как кольцо
        $A$ целостное, то $I^{s + 1}$ не имеет подмодуля кручения, следовательно, 
        $\tors{I^s \ox_A I} \subset \ker \mu$.

        Чтобы показать обратное включение, вычислим $\ker \mu$. Пусть $z \in I^s \ox_A I$. Тогда $z$
        имеет вид
        \begin{eqnarray*}
            z = a_0(x^s \ox x) &+& a_1(x^{s - 1}y \ox x) + \dots + a_s(y^s \ox x) + \\
                               &+& b_1(x^{s} \ox y) + \dots + b_s(xy^{s - 1} \ox y) + 
                               b_{s + 1}(y^s \ox y),\\
                               \text{где $a_i, b_i \in A$.}
        \end{eqnarray*}
        $\mu(z)$ будет иметь следующий вид:
        \begin{equation*}
            \mu(z) = a_0x^{s + 1} + (a_1 + b_1)x^sy + \dots + (a_s + b_s)xy^{s} + b_{s + 1}y^{s + 1}.
        \end{equation*}
        Приравняв $\mu(z) = 0$ и воспользовавшись тем фактом, что $x, y$ алгебраически независимы,
        мы получим условие на коэффициенты:
        \begin{equation*}
            \begin{cases}
                a_0 &= 0\\
                a_1 + b_1 &= 0\\
                \dots \\
                a_s + b_s &= 0\\
                b_{s + 1} &= 0.
            \end{cases}
        \end{equation*}
        Таким образом, $a_0 = b_{s + 1} = 0$, $a_i = -b_i$, $i = \bar{1, s}$ и 
        \begin{equation*}
            \ker \mu = \left< x^ny^{s - n} \ox x - x^{n + 1}y^{s - n - 1} \ox y |
            n = \bar{1, s-1}\right>_A.
        \end{equation*}
         
        Покажем, что образующая $\ker \mu$ вида $x^ny^{s - n} \ox x - x^{n + 1}y^{s - n - 1} \ox y$
        является элементом кручения. Рассмотрим выражение 
        $xy(x^ny^{s - n} \ox x - x^{n + 1}y^{s - n - 1} \ox y)$ и преобразуем его:
        \begin{multline*}
            xy(x^ny^{s - n} \ox x - x^{n + 1}y^{s - n - 1} \ox y) = \\
            x(x^ny^{s - n}) \ox xy - y(x^{n + 1}y^{s - n - 1}) \ox xy = \\
            x^{n + 1}y^{s - n} \ox xy - x^{n + 1}y^{s - n} \ox xy = 0.
        \end{multline*}
        Действительно, каждая образующая $\ker \mu$ является элементом кручения. Тем самым мы показали,
        что $\ker \mu \subset \tors{I^s \ox_A I}$.

        Таким образом,  мы показали, что $\tors{I^s \ox_A I} = \ker \mu $ и имеет место равенство
        \begin{equation*}
            \tors{I^s \ox_A I} = \left< x^ny^{s - n} \ox x - x^{n + 1}y^{s - n - 1} \ox y |
            n = \bar{1, s-1}\right>_A.
        \end{equation*}
    \end{Proof}

