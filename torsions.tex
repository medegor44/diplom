\section{Кручения в некоторых тензорных произведениях модулей}
    В задачах алгебраической геометрии, связанных с разрешением особенностей когерентных 
    алгебраических пучков, бывает необходимо 
    исследовать поведение когерентного алгебраического пучка при преобразованиях базисного 
    многообразия или схемы.
    Преобразование базисного многообразия подбирается так, чтобы трансформировать не локально 
    свободный когерентный пучок в локально
    свободный пучок на новом многообразии или схеме. 

    Локальным аналогом этой задачи является исследование свойств тензорного произведения модуля $M$ 
    над коммутативным кольцом $A$ на $A$-алгебру $\widetilde{A}$.

    В \cite{Timofeeva} автором изложена одна из возможных конструкций разрешения особенностей когерентного пучка, 
    локально сводящаяся к преобразованию $M \mapsto \widetilde{A} \otimes_A M$.
    Алгебра $\widetilde{A}$ получается при этом следующим образом: 
    $\widetilde{A} = \bigoplus_{s \geq 0} (I[t] + (t))^s / (t^{s + 1})$, где $I \subset A$ -- 
    ненулевой собственный
    идеал, $t$ -- элемент, трансцендентный над кольцом $A$.

    Рассмотрим коммутативное ассоциативное нетерово целостное кольцо $A$ с единицей. 
    
    \begin{Def}
        Пусть $I \subset A$ --- идеал. \textit{Алгебра раздутия} идеала $I$ задается выражением 
        $$\widehat{A} := \bigoplus_{s \geq 0} I^s.$$
        При этом сложение и умножение элементов кольца $\widehat{A}$ и действие элементов кольца $A$
        на элементы кольца $\widehat{A}$ наследуются с операций кольца $A$.
    \end{Def}
    Кольцо $\widehat{A}$ градуировано, то есть, если $x \in I^s, y \in I^t$, то $xy \in I^{s + t}$.
    Если $x \in I^s$, то будем говорить, что $x$ имеет степень $s$.
    Также отметим, следующее: если кольцо $A$ --- целостное, то и кольцо $\hat A$ тоже будет целостным.

    \begin{Def}
        Пусть $M$ -- произвольный $A$-модуль, $A$ --- целостное кольцо. 
        \textit{Кручением} $\Tors{A}{M}$ называется множество
        \begin{equation*}
            \Tors{A}{M} = \left\{x \in M | \exists a \in A\setminus 0 : ax = 0 \right\}.
        \end{equation*}
    \end{Def}
    \begin{Def}
        Будем говорить, что $A$-модуль $M$ является \textit{модулем без кручения}, если $\Tors{A}{M} = 0$.
    \end{Def}

    Заметим, если $A$ не является целостным кольцом, то $\Tors{A}{M}$ не обязательно является подмодулем
    в $M$.

    Далее в тексте под $\tors{M}$ без нижнего индекса будем подразумевать $\Tors{A}{M}$.

    Пусть $M$ -- $A$-модуль без кручения. Поскольку тензорное произведение не является точным слева, 
    при тензорном умножении $M$ на алгебру раздутия $\widehat{A}$ в модуле $\widehat{A} \ox_A M$ 
    может возникнуть кручение. 

    Решается следующая частная задача: описать подмодуль кручения $\tors{\widehat{A} \ox_A I}$ 
    $A$-модуля $\widehat{A} \ox_A I$.

    Пусть, для простоты, идеал $I = (x, y)$ порожден элементами $x, y \in A$. Выясним, как устроены
    его степени. 

    \begin{Theorem}
        Пусть $s \geq 1$, тогда $I^s = (x^s, x^{s-1}y, \dots, xy^{s-1}, y^s)$.
    \end{Theorem}
    \begin{Proof}
        Действуем методом математической индукции. Пусть $s = 1$. Тогда $I^1 = (x, y)$ -- верно. Пусть утверждение
        верно для значений $s \leq r$. При $s = r + 1$ имеем:
        \begin{equation*}
            I^{r + 1} = I^{r}I = \left\{\left(\sum_{n = 0}^{r} a_nx^ny^{n - r}\right)(b_1x + b_0y) 
            \Bigg| a_n, b_m \in A, n = \bar{0, r}, m = \bar{0, 1}\right\}.
        \end{equation*}
        Теперь, раскрывая скобки, получим
        \begin{multline*}
            I^{r + 1} = \{ b_0a_0y^{r + 1} + (b_1a_0 + b_0a_1)xy^{r} + \dots \\+ 
                (b_1a_{r - 1} + b_0a_{r})x^ry + b_1a_rx^{r + 1}  | a_n, b_m \in A, n = \bar{0, r}, m = \bar{0, 1} \}
        \end{multline*}
        Таким образом, в силу произвольности коэффициентов $a_i, b_j$,
        \begin{equation*}
            I^{r + 1} = (x^{r + 1}, x^ry, \dots, xy^r, y^{r + 1}),
        \end{equation*}
        что завершает доказательство теоремы.
    \end{Proof}

    Так как тензорное произведение дистрибутивно относительно прямой суммы,
    то справедлива цепочка равенств:

    \begin{equation*}
        \widehat{A} \ox_A I = \left(\bigoplus_{s \geq 0}{I^s}\right) \ox_A I = 
            \bigoplus_{s \geq 0}{\left(I^s \ox_A I\right)}.
    \end{equation*}

    \begin{Proposal} \label{torsIdentity}
        Пусть $\{M_j | j \in J\}$ -- семейство $A$-модулей, и кольцо $A$ -- целостное. Тогда
        \begin{equation*}
            \tors{\bigoplus_{j \in J} M_j} = \bigoplus_{j \in J} \tors{M_j}.
        \end{equation*}
    \end{Proposal}
    \begin{Proof}
        Покажем, что $\tors{\bigoplus_{j \in J} M_j} \subset \bigoplus_{j \in J} \tors{M_j}$. 
        Пусть \linebreak $t \in \tors{\bigoplus_{j \in J} M_j}$. По определению, существует такое 
        $a \in A \setminus 0$, что $at = 0$. Заметим, что $t = (t_0, t_1, \dots, t_j, \dots)$, где 
        только конечное число компонент $t_j$ отлично от нуля. Так как умножение на элементы прямой 
        суммы производится покомпонентно, то 
        \begin{equation*}
            at = (at_0, at_1, \dots, at_j, \dots) = 0,
        \end{equation*}
        из чего следует, что 
        \begin{equation*}
            at_1 = at_0 = \dots = at_j = \dots = 0
        \end{equation*}
        и $t_0 \in \tors{M_0}$, $t_1 \in \tors{M_1}$, \dots, $t_j \in \tors{M_j}$, \dots . 
        Таким образом,\linebreak $t \in \bigoplus_{j \in J} \tors{M_j}$. Теперь докажем обратное включение.
        Пусть $t \in \bigoplus_{j \in J} \tors{M_j}$. Пусть $t_{i_1}, t_{i_2}, \dots, t_{i_k}$ 
        -- все компоненты $t$, отличные от нуля. Как отмечалось ранее, их будет конечное число. По определению, 
        найдутся  $a_{i_1}, a_{i_2}, \dots, a_{i_k} \in A$ все отличные от нуля и такие, 
        что $a_{i_1}t_{i_1} = a_{i_2}t_{i_2} = \dots = a_{i_k}t_{i_k} = 0$. Обозначим 
        $a := a_{i_1}a_{i_2}\dots a_{i_k}$. Так как кольцо $A$ целостное, то ни при каких отличных от нуля 
        $a_{i_l}$ их произведение не будет равно нулю. Тогда
        \begin{equation*}
            at_{i_l} = (a_{i_1}\dots a_{i_{l-1}}a_{i_{l + 1}}\dots a_{i_k})a_{i_l}t_{i_l} = 0,
        \end{equation*}
        что справедливо для всех $l = \bar{1,k}$. Тем самым мы показали, 
        что существует такое $a \in A \setminus 0$, что $at = 0$. Значит 
        $t \in \tors{\bigoplus_{j \in J} M_j}$.
    \end{Proof}

    Теперь, воспользовавшись предложением \ref{torsIdentity}, можно записать следующее:
    \begin{equation*}
        \tors{\bigoplus_{s \geq 0} \left( I^s \ox_A I \right)} = 
        \bigoplus_{s \geq 0}\tors{ I^s \ox_A I }.
    \end{equation*}
    Таким образом, исходная задача свелась к вычислению подмодуля кручения \linebreak$\tors{ I^s \ox_A I }$ 
    $A$-модуля $I^s \ox_A I$.
    \begin{Theorem} \label{tors_simple}
        Пусть образующие иделала $I = (x, y)$, имеющие равные степени, алгебраически независимы. 
        Тогда $\tors{I^s \ox_A I}$ описывается следующим образом:
        \begin{equation*}
            \tors{I^s \ox_A I} = \left< x^ny^{s - n} \ox x - x^{n + 1}y^{s - n - 1} \ox y |
             n = \bar{1, s-1}\right>_A.
        \end{equation*}
    \end{Theorem}
    \begin{Proof}
        Так как идеалы $I^s$ и $I$ являются ко\-неч\-но по\-ро\-жден\-ны\-ми $A$-мо\-ду\-ля\-ми, то, воспользовавшись свойством 
        тензорного произведения для двух конечно порожденных модулей, имеем
        \begin{equation*}
            I^s \ox_A I = \left< x^ny^{s - n} \ox x, x^ny^{s - n} \ox y | n = \bar{0, s} \right>_A.
        \end{equation*}
        
        Пусть $\mu : I^s \ox_A I \rightarrow I^{s + 1}$ -- гомоморфизм, который действует на образующих
        следующим образом: $x^ny^{s - n} \ox x \mapsto x^{n + 1}y^{s - n}$, 
        $x^ny^{s - n} \ox y \mapsto x^ny^{s - n + 1}$.
        Докажем, что $\ker \mu = \tors{I^s \ox_A I}$. Очевидно, что этот гомоморфизм сюръективен. Тогда,
        согласно теореме о гомоморфизме, $I^{s + 1} \simeq (I^s \ox_A I) / \ker \mu$. Так как кольцо
        $A$ целостное, то $I^{s + 1}$ не имеет подмодуля кручения, следовательно, 
        $\tors{I^s \ox_A I} \subset \ker \mu$.

        Чтобы показать обратное включение, вычислим $\ker \mu$. Пусть $z \in I^s \ox_A I$, тогда $z$
        имеет вид
        \begin{eqnarray*}
            z = a_0(x^s \ox x) &+& a_1(x^{s - 1}y \ox x) + \dots + a_s(y^s \ox x) + \\
                               &+& b_1(x^{s} \ox y) + \dots + b_s(xy^{s - 1} \ox y) + 
                               b_{s + 1}(y^s \ox y),
        \end{eqnarray*}
        где $a_i, b_i \in A$. Тогда $\mu(z)$ будет иметь следующий вид:
        \begin{equation*}
            \mu(z) = a_0x^{s + 1} + (a_1 + b_1)x^sy + \dots + (a_s + b_s)xy^{s} + b_{s + 1}y^{s + 1}.
        \end{equation*}
        Приравняв $\mu(z) = 0$ и воспользовавшись тем фактом, что $x, y$ алгебраически независимы,
        мы получим условия на коэффициенты:
        \begin{equation*}
            \begin{cases}
                a_0 &= 0,\\
                a_1 + b_1 &= 0,\\
                \dots \\
                a_s + b_s &= 0,\\
                b_{s + 1} &= 0.
            \end{cases}
        \end{equation*}
        Отсюда, $a_0 = b_{s + 1} = 0$, $a_i = -b_i$, $i = \bar{1, s}$ и 
        \begin{equation*}
            \ker \mu = \left< x^ny^{s - n} \ox x - x^{n + 1}y^{s - n - 1} \ox y |
            n = \bar{1, s-1}\right>_A.
        \end{equation*}
         
        Покажем, что любая образующая $\ker \mu$, то есть $x^ny^{s - n} \ox x - x^{n + 1}y^{s - n - 1} \ox y$,
        является элементом кручения. Рассмотрим выражение 
        $xy(x^ny^{s - n} \ox x - x^{n + 1}y^{s - n - 1} \ox y)$ и преобразуем его:
        \begin{multline*}
            xy(x^ny^{s - n} \ox x - x^{n + 1}y^{s - n - 1} \ox y) = \\
            x(x^ny^{s - n}) \ox xy - y(x^{n + 1}y^{s - n - 1}) \ox xy = \\
            x^{n + 1}y^{s - n} \ox xy - x^{n + 1}y^{s - n} \ox xy = 0.
        \end{multline*}
        Действительно, каждая образующая $\ker \mu$ является элементом кручения. Тем самым мы показали
        включнение $\ker \mu \subset \tors{I^s \ox_A I}$.

        Таким образом,  мы доказали, что $\tors{I^s \ox_A I} = \ker \mu $, и имеет место равенство
        \begin{equation*}
            \tors{I^s \ox_A I} = \left< x^ny^{s - n} \ox x - x^{n + 1}y^{s - n - 1} \ox y |
            n = \bar{1, s-1}\right>_A.
        \end{equation*}
    \end{Proof}

    Результат данной теоремы можно обобщить следующим образом.
    \begin{Theorem}
        Пусть образующие идеала $I = (x, y)$ алгебраически независимы. Тогда подмодуль кручения
        $\tors{I^s \ox_A I^r}$ описывается следующим образом:
        \begin{equation*}
            \tors{I^s \ox_A I^r} = 
                \left\{ \sum_{\substack{0 \le n \le s \\ 0 \le m \le r}}a_{nm}x^ny^{s - n} 
                    \ox x^my^{r - m}  \right\}, 
        \end{equation*}
        где коэффициенты $a_{ij}$ удовлетворяют соотношению 
        $$\sum_{i + j = n + m} a_{ij} = 0\;\text{для всех n, m}.$$
    \end{Theorem}
    \begin{Proof}
        Доказательство проводится по схеме, аналогичной доказательству теоремы \ref{tors_simple}.
        Модуль $I^s \ox_A I^r$ имеет вид
        \begin{equation*}
            I^s \ox_A I^r = \left< x^ny^{s - n} \ox x^my^{r - m} | 
                n = \bar{0, s}, m = \bar{0, m} \right>_A.
        \end{equation*}
        Рассмотрим гомоморфизм $\mu : I^s \ox_A I^r \rightarrow I^{s + r}$, который действует на 
        образующих как $x^ny^{s - n} \ox x^my^{r - m} \mapsto x^{n + m}y^{s + r - n - m}$. 
        Докажем, что $\ker \mu = \tors{I^s \ox_A I^r}$. Очевидно, что $\mu$ сюръективен и, 
        воспользовавшись теоремой о гомоморфизме, мы можем записать 
        $I^{s + r} \simeq (I^s \ox_A I^r) / \ker \mu$. Так как кольцо $A$ целостное, то $I^{s + r}$
        является модулем без кручения, из чего следует, что $\tors{I^s \ox I^r} \subset \ker \mu$.
        
        Покажем обратное включение. Для этого вычислим $\ker \mu$. Любой элемент $z \in I^s \ox_A I^r$
        записывается в виде линейной комбинации образующих 
        \begin{equation*}
            z = \sum_{\substack{0 \le n \le s \\ 0 \le m \le r}}a_{nm}x^ny^{s - n} \ox x^my^{r - m},
        \end{equation*} 
        где $a_{nm} \in A$.
        Вычислив $\mu(z)$, получим следующее
        \begin{equation*}
            \mu(z) = \sum_{\substack{0 \le n \le s \\ 0 \le m \le r}}a_{nm}x^{n + m}y^{s + r - n - m}.
        \end{equation*}
        Сгруппируем слагаемые с одинаковыми степенями $x$ и тогда полученное выражение запишется в виде
        \begin{equation*}
            \mu(z) = \sum_{k = 0}^{s + r}\left( \sum_{i + j = k} a_{ij} \right)x^ky^{s + r - k}.
        \end{equation*}

        Так как образующие алгебраически независимы, то из равенства $\mu(z) = 0$ следует, что
        \begin{equation*}
            \sum_{i + j = k} a_{ij} = 0.
        \end{equation*}
        С учетом полученного соотношения, элементы ядра имеют вид
        \begin{equation} \label{kerel}
            z = \sum_{k=0}^{s + r} \sum_{n = 0}^{\min(s, k)} a_{n,k-n}x^ny^{s - n} \ox x^{k - n}y^{r - k + n},
        \end{equation}
        докажем, что $z \in \tors{I^s \ox_A I^r}$. Действительно, зафиксируем $k$, $n \leq \min(s, k)$.
        Рассмотрим образующую $x^ny^{s - n} \ox x^{k - n}y^{r - k + n}$ и умножим ее на 
        $x^ry^r$, где $r$ --- показатель степени идеала $I^r$. Имеем
        \begin{multline*}
            x^ry^r(x^ny^{s - n} \ox x^{k - n}y^{r - k + n}) = \\
            x^{k - n}y^{r - (k - n)}x^ny^{s - n} \ox x^{r - (k - n)}y^{k - n}x^{k - n}y^{r - k + n} = \\
            x^ky^{r + s - k} \ox x^ry^r.
        \end{multline*}
        Умножив выражение \eqref{kerel} на $x^ry^r$,  мы получим сумму следующего вида
        \begin{equation*}
            x^ry^r\sum_{k=0}^{s + r}\sum_{n = 0}^{\min(s, k)} a_{n,k-n}x^ny^{s - n} \ox x^{k - n}y^{r - k + n} =
            \sum_{k=0}^{s + r} \left[ \left(\sum_{n = 0}^{\min(s, k)}a_{n,k-n}\right)x^ky^{r + s - k} \ox x^ry^r \right] = 0,
        \end{equation*}
        где последнее равенство следует из условия, наложенного на коэффициенты $a_{ij}$. 
        Данное равенство выполнено при всех $k = \bar{0, s + r}$.
        Таким образом, мы доказали, что $\ker \mu \subset \tors{I^s \ox_A I^r}$.
    \end{Proof}
    \begin{Corollary}
        Пусть числа $a, b$ -- натуральные, $I = (x, y)^a$, $J = (x, y)^b$, тогда
        \begin{equation*}
            \tors{I^s \ox_A J} = 
                \left\{ \sum_{\substack{0 \le n \le as \\ 0 \le m \le b}}a_{nm}x^ny^{as - n} \ox x^my^{b - m} \right\},
        \end{equation*}
        где коэффициенты $a_{ij}$ удовлетворяют соотношению
        $$\sum_{i + j = n + m} a_{ij} = 0\;\text{для всех n, m}.$$
    \end{Corollary}
    Заметим, что если на прямой сумме $\bigoplus_{s \ge 0} I^s$ рассмотреть покомпонентное умножение
    (вместо структуры градуированного кольца), то полученные нами результаты не изменятся.

    Исходную задачу можно видоизменить, заменив алгебру раздутия на алгебру
    \begin{equation*}
        \widetilde{A} := \bigoplus_{s \geq 0}{(I[t] + (t))^s / (t^{s + 1})},
    \end{equation*}
    где $t$ -- элемент, трансцендентный над $A$, а умножение определяется покомпонентно. 
    Отметим, что алгебра $\widetilde{A}$ является
    $A$-алгеброй без кручения, однако, если рассматривать $\widetilde{A}$ как алгебру над 
    $\widetilde{A}$, то возникают элементы кручения, например, $(0, t, 0, \dots)$. Далее будем 
    работать с $\widetilde{A}$ как с $A$-алгеброй.

    Обозначим $s$-ое слагаемое в прямой сумме как $I^s_t := (I[t] + (t))^s / (t^{s + 1})$.
    Сформулируем вспомогательную теорему
    \begin{Lemma}
        $A$-модуль $I^s_t$ допускает следующее разложение в сумму своих $A$-под\-мо\-ду\-лей
        \begin{equation} \label{sl}
            I^s_t = \left< 1  \right>_{I^s} + 
                    \left< t \right>_{I^{s-1}} + 
                    \dots +
                    \left< t^{s-1} \right>_{I} + 
                    \left< t^s \right>_A.
        \end{equation}
    \end{Lemma}
    \begin{Proof}
        Сразу отметим, что при вычислении $I^s_t$ будем рассматривать многочлены степени не больше $s$,
        так как при факторизации по $(t^{s + 1})$ большие степени обратятся в 0. 
        По определению, $(I[t] + (t))^s$ состоит из произведений $s$ произвольных
        элементов $I[t] + (t)$. Поэтому, чтобы выяснить структуру $(I[t] + (t))^s$, необходимо рассмотреть
        произведение
        \begin{equation*}
            \prod_{n = 1}^s \left( a_{n0} + (a_{n1} + b_n)t + a_{n2}t^2 + \dots + a_{ns}t^s \right),
        \end{equation*}
        где $a_{nj} \in I, b_n \in A$, $n = \bar{1, s}, j = \bar{0, s}$.
        Выясним, к каким степеням идеала $I$ принадлежат коэффициенты при $t^k$, $0 \leq k \leq s$.
        Рассмотрим слагаемые в коэффициенте при $t^k$, которые имеют вид
        \begin{equation*}
            b_{j_1}b_{j_2}\dots  b_{j_k}a_{j_{k+1}0}\dots a_{j_s0},
        \end{equation*}
        где множества $\{j_1, \dots, j_k\}, \{j_{k+1}, \dots, j_s\} \subset \{1, \dots, s\}$ не пересекаются, а
        $\{j_1, \dots, j_s\} = \{1, \dots, s\}$. Очевидно, что это слагаемое принадлежит $I^{s - k}$,
        при этом взять в произведении большее число множителей, необязательно принадлежащих идеалу $I$, 
        нельзя, так как мы ограничены степенью $k$. Поэтому
        $I^{s - k}$ является наименьшей степенью идеала, к которой могут принадлежать слагаемые в коэффициенте при $t^k$.
        Однако, отметим, что для любой степени идеала $I^r$, где $r \geq s - k$ найдется такое слагаемое 
        в коэффициенте при $t^k$, что оно принадлежит $I^r$, например, пусть $r = l + (s - k)$
        \begin{equation*}
            a_{11}a_{21}\dots a_{l1}b_{l + 1}\dots b_k a_{k + 1, 0}\dots a_{s0} \in I^r.
        \end{equation*}

        Так как все коэффициенты были произвольные, то имеет место разложение $I^s_t$ как $A$-модуля 
        в сумму своих $A$-подмодулей
        \begin{equation*}
            I^s_t = \left< 1, t, \dots, t^s \right>_{I^s} + 
            \left< t, t^2, \dots, t^s \right>_{I^{s-1}} + 
            \dots +
            \left< t^{s-1}, t^s \right>_{I} + 
            \left< t^s \right>_A.
        \end{equation*}
        Заметим, так как справедливы включения $I^s \subset I^{s - 1} \subset \dots \subset I \subset A$,
        то справедливы включения $\left< t^k \right>_{I^s} \subset \left< t^k \right>_{I^{s - k}}$. 
        Поэтому исходное разложение можно переписать в виде
        \begin{equation*}
            I^s_t = \left< 1  \right>_{I^s} + 
                    \left< t \right>_{I^{s-1}} + 
                    \dots +
                    \left< t^{s-1} \right>_{I} + 
                    \left< t^s \right>_A.
        \end{equation*}
    \end{Proof}
    Заметим, что сумма \eqref{sl} является прямой внутренней суммой своих подмодулей.
    Теперь, зная строение $A$-модуля $I^s_t$, можно сформулировать теорему
    \begin{Theorem} \label{tors_hard}
        Пусть $J \subset A$ -- идеал в $A$, тогда
        \begin{equation*}
            \tors{I^s_t \ox_A J} = t^0\tors{I^s \ox_A J} + t^1\tors{I^{s-1}\ox_A J} + \dots + t^{s - 1}\tors{I \ox_A J}.
        \end{equation*}
        В частности,
        \begin{equation*}
            \tors{I^s_t \ox_A I} = t^0\tors{I^s \ox_A I} + t^1\tors{I^{s-1}\ox_A I} + \dots + t^{s - 1}\tors{I \ox_A I}.
        \end{equation*}
    \end{Theorem}
    \begin{Proof}
        Так как тензорное произведение дистрибутивно относительно прямой суммы и, в силу теоремы 
        \ref{torsIdentity}, можно записать 
        \begin{multline*}
            \tors{I^s_t \ox_A J} =  \tors{\left< t^0 \right>_{I^s} \ox_A J} + 
            \tors{\left< t^1 \right>_{I^{s-1}} \ox_A J} + \dots \\+
            \tors{\left< t^{s - 1} \right>_{I^1} \ox_A J} + 
            \tors{\left< t^{s} \right>_{A} \ox_A J}.
        \end{multline*}
        Так как $t$ -- элемент, трансцендентный над $A$, то его не аннулирует никакой многочлен с
        коэффициентами из $A$. Значит, он не даст вклада в кручение и его можно вынести за знак $\tors{\cdot}$.
        Таким образом имеем
        \begin{multline*}
            \tors{I^s_t \ox_A J} =  t^0\tors{\left< 1 \right>_{I^s} \ox_A J} + 
            t^1 \tors{\left< 1\right>_{I^{s-1}} \ox_A J} + \dots \\+
            t^{s - 1}\tors{\left< 1 \right>_{I^1} \ox_A J} + 
            t^{s}\tors{\left< 1 \right>_{A} \ox_A J}.
        \end{multline*}
        Но $\left< 1 \right>_{I^k}$, очевидно, является самим идеалом $I^k$. Таким образом, имеем 
        \begin{equation*}
            \tors{I^s_t \ox_A J} = t^0\tors{I^s \ox_A J} + t^1\tors{I^{s-1}\ox_A J} + \dots + t^{s - 1}\tors{I \ox_A J}.
        \end{equation*}
    \end{Proof}
    Задача свелась к вычислению $\tors{I^s \ox J}$. Пусть $J = I$, тогда справедлива следующая 

    \begin{Theorem}
        Пусть образующие идеала $I$ алгебраически независимы, тогда кручение $A$-модуля $I^s_t \ox_A I$ дается суммой своих подмодулей:
        \begin{equation*}
            \begin{split}
                \tors{I^s_t \ox_A I} = \left<x^{s-1}y \ox x - x^s\ox y, x^{s-2}y^2\ox x - x^{s-1}y\ox y, \dots , y^s \ox x - xy^{s - 1} \ox y\right>_A + \\
                t\left<x^{s-2}y \ox x - x^{s-1}\ox y, x^{s-3}y^2\ox x - x^{s-2}y\ox y, \dots , y^{s-1} \ox x - xy^{s - 2} \ox y\right>_A + \\
                \dots + \\
                t^{s-1}\left<x\ox y - y\ox x\right>_A.
            \end{split}
        \end{equation*}
    \end{Theorem}
    \begin{Proof}
        Воспользуемся теоремой \ref{tors_hard} и для каждого $\tors{I^s \ox_A I}$ применим теорему
        \ref{tors_simple}.
    \end{Proof}

    Как было отмечено ранее, $\widetilde{A}$ является алгеброй с кручением как алгебра над $\widetilde{A}$
    с покомпонентным умножением. 
    Выясним, какой вид имеет $\Tors{\widetilde{A}}{\widetilde{A}}$. Заметим следующее
    \begin{equation*}
        \Tors{\widetilde{A}}{\widetilde{A}} = \Tors{\bigoplus I^s_t}{\bigoplus I^s_t}  = \bigoplus{\Tors{I^s_t}{I^s_t}},
    \end{equation*}
    так как умножение в прямой сумме осуществляется покомпонентно. Таким образом, мы свели исходную
    задачу к следующей: описать $\Tors{I^s_t}{I^s_t}$. Справедлива
    \begin{Theorem}
        \begin{equation*}
            \Tors{I^s_t}{I^s_t} = 
            \left< t \right>_{I^{s-1}} + 
            \dots +
            \left< t^{s-1} \right>_{I} + 
            \left< t^s \right>_A.
        \end{equation*}
    \end{Theorem}
    \begin{Proof}
        Рассмотрим элемент $I^s_t$ следующего вида 
        \begin{equation} \label{rightel}
            a_1t + a_2t^2 + \dots + a_st^s,
        \end{equation}
        где $a_i \in I^{s - i}$, и умножим его на $1\cdot t^s \neq 0$.
        \begin{equation*}
            (a_1t + a_2t^2 + \dots + a_st^s)t^s = a_1t^{s + 1} + a_2t^{s + 2} + \dots + a_st^{2s} = 0,
        \end{equation*}
        то есть, мы показали, что элементы вида \eqref{rightel} действительно являются элементами кручения.
        Покажем, что никакие другие элементы вклада в кручение не дадут. Предположим, что
        \begin{equation*} 
            f = a_0 + a_1t + a_2t^2 + \dots + a_st^s \in \Tors{I^s_t}{I^s_t},
        \end{equation*}
        где $a_0 \neq 0$. По определению, существует такой элемент $g \in I^s_t \setminus 0$, что
        $fg = 0.$
        Пусть 
        \begin{equation*}
            g = b_0 + b_1t + \dots + b_st^s \neq 0.
        \end{equation*}
        Рассмотрим коэффициенты при $t^k$, $k = \bar{0, s}$ в произведении $fg$. Коэффициент при $t^k$ обозначим как $[t^k]$.
        \begin{align*}
            [t^0] &= a_0b_0 = 0\\
            [t^1] &= a_0b_1 + a_1b_0 = 0\\
            \vdots\\
            [t^s] &= a_0b^s + \dots + a_{s-1}b_1 + a_sb_0 = 0.
        \end{align*}
        Так как кольцо целостное, $a_0 \neq 0$, то, из уравнения на $[t^0]$, получаем $b_0 = 0$. Подставив
        $b_0 = 0$ в уравнение на $[t^1]$ и воспользовавшись целостностью кольца, получим $b_1 = 0$. Повторяя 
        эти рассуждения далее, получим, что $b_0 = b_1 = \dots = b_s = 0$. Таким образом, $f$ аннулирует 
        только 0, значит $f \not \in \Tors{I^s_t}{I^s_t}$. 

        Таким образом, действительно, только элементы вида \eqref{rightel} являются элементами кручения.
        Все такие элементы описываются суммой 
        \begin{equation*}
            \left< t \right>_{I^{s-1}} + 
            \dots +
            \left< t^{s-1} \right>_{I} + 
            \left< t^s \right>_A.
        \end{equation*}
    \end{Proof}

    Рассмотрим следующую задачу. Описать кручение $\hat A$-модуля $M \ox_A \widehat{A}$, если 
    $A$-модуль $M$ включается в короткую точную последовательность вида
    \begin{equation*}
        0 \rightarrow I_1 \xrightarrow{i} M \xrightarrow{\epsilon} I_2 \rightarrow 0,
    \end{equation*}
    где $I_1, I_2 \subset A$ --- идеалы в кольце $A$, $A$ --- целостное, нетерово кольцо. 

    Обозначим $\widehat{M} := M \ox_A \widehat{A}$. Так как тензорное произведение не точно слева, то
    имеем последовательность вида
    \begin{equation*}
        \hat I_1 \xrightarrow{\hat i} \hat M \xrightarrow{\hat \epsilon} \hat I_2 \rightarrow 0,
    \end{equation*}
    в которой $\hat i := i \ox 1$, $\hat \epsilon := \epsilon \ox 1$. Пусть $\tau := \ker \hat i$. Тогда
    получим точную последовательность 
    \begin{equation*}
        0 \rightarrow \tau \rightarrow \hat I_1 \xrightarrow{\hat i} \hat M \xrightarrow{\hat \epsilon} \hat I_2 \rightarrow 0.
    \end{equation*}
    Справедливы вложения
    $$   
        \xymatrix{
            0 \ar[r] 
            &\tau \ar[r] 
            &\hat I_1 \ar[r]^{\hat i} 
            &\hat M \ar[r]^{\hat \epsilon} 
            &\hat I_2 \ar[r] 
            &0\\
            0 \ar[r] 
            & \Tors{\hat A}{\tau} \ar[r] \ar@{^{(}->}[u]
            & \Tors{\hat A}{\hat I_1} \ar[r]^{\hat i'} \ar@{^{(}->}[u]
            & \Tors{\hat A}{\hat M} \ar[r]^{\hat \epsilon'} \ar@{^{(}->}[u]
            & \Tors{\hat A}{\hat I_2} \ar[r] \ar@{^{(}->}[u]
            &0
        }
    $$
    где гомоморфизмы в нижней строке получены путем ограничения гомоморфизмов верхней строки на соответствующие множества. 
    Разложим гомоморфизм $\hat i'$ в композицию сюръективного и инъективного гомоморфизмов и рассмотрим нижнюю строку
    $$   
        \xymatrix{
            0 \ar[r] 
            & \Tors{\hat A}{\tau} \ar[r] 
            & \Tors{\hat A}{\hat I_1} \ar[r]^{\hat i'} \ar@{>>}[rd]
            & \Tors{\hat A}{\hat M} \ar[r]^{\hat \epsilon'} 
            & \Tors{\hat A}{\hat I_2} \ar[r] 
            &0
            \\
            &
            &
            &\frac{\Tors{\hat A}{\hat I_1}}{\Tors{\hat A}{\tau}} \ar@{^{(}->}[u] 
        }
    $$
    Отметим, что $\hat I_1, \hat I_2$ конечно порождены, согласно предложению 2.17 книги \cite{A-M}, 
    как $\hat A$-модули, поэтому $\Tors{\hat A}{\hat I_2}$
    конечно порожден как подмодуль нетерова модуля, $\frac{\Tors{\hat A}{\hat I_1}}{\Tors{\hat A}{\tau}}$
    конечно порожден как образ конечно порожденного модуля.
    Поэтому мы можем воспользоваться предложением 4 $\S 4$ гл. 1 книги 
    \cite{Zulanke}, которое утверждает, что расширение последовательности 
    \begin{equation*}
        0 \rightarrow \frac{\Tors{\hat A}{\hat I_1}}{\Tors{\hat A}{\tau}} \rightarrow 
        \Tors{\hat A}{\hat M} \xrightarrow{\hat \epsilon'} \Tors{\hat A}{\hat I_2} \rightarrow 0
    \end{equation*}
    порождено образами порождающих ядра и прообразами порождающих коядра последовательности. 
    Пусть $\hat I_2 = \left< \bar{z_1}, \bar{z_2}, \dots , \bar{z_m} \right>_{\hat A}$, $z_i \in \Tors{\hat A}{\hat M$} 
    --- произвольно выбранный прообраз $\bar{z_i}$ $(i =\bar{1, m})$ и 
    $\frac{\Tors{\hat A}{\hat I_1}}{\Tors{\hat A}{\tau}} = \left< x_1, x_2, \dots, x_n\right>_{\hat A}$,
    $\bar{x_j}$ --- образ $x_j$ в $\Tors{\hat A}{\hat M}$ $(j=\bar{1, n})$,  тогда
    \begin{equation} \label{torsHatM}
        \Tors{\hat A}{\hat M}= \left< \bar x_1, \bar x_2, \dots , \bar x_n \right>_{\hat A} + 
        \left< z_1, z_2, \dots , z_m \right>_{\hat A}.
    \end{equation}

    Теперь выясним как охарактеризовать кручение произвольного $\hat A$-модуля \linebreak $M \ox_A \hat A$, при 
    условии что $M$ --- нетеров $A$-модуль. Для этого нам потребуется утверждение:
    \begin{Theorem} \label{Mvee1}
        $M^{\vee} := \Hom_A(M, A)$ --- модуль без кручения.
    \end{Theorem}
    \begin{Proof}
        Известно, что любой конечно порожденный модуль является образом свободного модуля подходящего
        ранга \cite{Zulanke}, то есть точна тройка
        \begin{equation*}
            0 \rightarrow K \rightarrow A^n \rightarrow M \rightarrow 0,
        \end{equation*}
        где $K = \ker(A^n \rightarrow M)$.
        Перейдем от нее к двойственной. Получим последовательность
        \begin{equation*}
            0 \rightarrow M^\vee \rightarrow A^n \rightarrow \dots
        \end{equation*}
        Так как $A$ --- целостное, то $A^n$ --- модуль без кручения, 
        $M^\vee$ обладает вложением в $A^n$, следовательно, $M^\vee$ тоже модуль без кручения.
    \end{Proof}

    Пусть $m \in M^{\vee} \setminus 0$. Рассмотрим гомоморфизм $A \rightarrow M^\vee$, $\alpha \mapsto \alpha m$.
    Заметим, что этот гомоморфизм инъективен, так как в противном случае $m$ являлся бы элементом 
    кручения, что невозможно по теореме \ref{Mvee1}. Имеем точную тройку $A$-модулей:
    \begin{equation*}
        0 \rightarrow A \rightarrow M^\vee \rightarrow N \rightarrow 0.
    \end{equation*}
    Перейдя от нее к двойственной, получим последовательность, не являющуюся точной справа
    \begin{equation*}
        0 \rightarrow N^\vee \rightarrow M^{\vee \vee} \rightarrow A \rightarrow \dots
    \end{equation*}
    Разложим гомоморфизм $M^{\vee \vee} \rightarrow A$ в композицию сюръективного и инъективного 
    гомоморфизмов
    \begin{equation*}
        \xymatrix{
            0 \ar[r]
            &N^\vee \ar[r] 
            &M^{\vee \vee} \ar[r] \ar@{>>}[d]
            &A \ar[r]
            &\dots \\
            &
            &
            M^{\vee \vee} / N^\vee \ar@{^{(}->}[ru]
        }
    \end{equation*}
    Так как $ M^{\vee \vee} / N^\vee$ обладает вложением в $A$ как $A$-модуль, то имеет место 
    изоморфизм  $ M^{\vee \vee} / N^\vee \simeq J \subset A$ --- некоторый идеал в кольце $A$. Таким 
    образом имеем новую точную тройку
    \begin{equation*}
        0 \rightarrow N^\vee \rightarrow M^{\vee \vee} \rightarrow J \rightarrow 0.
    \end{equation*}
    Поскольку $M$ --- $A$-модуль без кручения, то имеет место вложение \linebreak
    $ M \lhook\joinrel\xrightarrow{\epsilon} M^{\vee \vee} : s \mapsto \epsilon_s$, где $\epsilon_s : t \mapsto t(s)$, 
    включаемое в диаграмму
    \begin{equation*}
        \xymatrix{
            0 \ar[r]
            &N^\vee \ar[r] 
            &M^{\vee \vee} \ar[r] 
            &J \ar[r]
            &0\\
            &
            &M \ar@{^{(}->}[u]
        }
    \end{equation*}
    Обозначим $M_1 := \ker(M \rightarrow J_1)$, $J_1 := \im(M \hookrightarrow M^{\vee \vee} \rightarrow J)$ --- 
    идеал в $A$, при этом выполнены вложения $J_1 \subset J \subset A$. 
    Имеем диаграмму с точными строками
    \begin{equation*}
        \xymatrix{
            0 \ar[r]
            &N^\vee \ar[r] 
            &M^{\vee \vee} \ar[r] 
            &J \ar[r]
            &0\\
            0 \ar[r]
            &M_1 \ar@{^{(}->}[u]\ar[r]
            &M \ar@{^{(}->}[u]\ar[r]
            &J_1 \ar@{^{(}->}[u]\ar[r]
            &0
        }
    \end{equation*}
    Так как $M_1 \subset M$, $M$ --- нетеров, следовательно, $M_1$ тоже нетеров. Повторим эти же
    действия для $M_1$, потом для $M_2$ и так далее. Имеем убывающую фильтрацию
    \begin{equation} \label{subsets}
        \dots \subset M_2 \subset M_1 \subset M.
    \end{equation}
    Так как нетеров модуль необязательно артинов, то эта последовательность может быть бесконечной. Покажем 
    что в нашем случае это не так и цепочка будет обрываться. Перейдем к локализации в нулевом идеале кольца $A$. 
    $A \hookrightarrow A_0 =: Q(A)$ --- поле частных кольца $A$.
    По свойству точности локализации имеем точную тройку $A_0$-векторных пространств
    \begin{equation*}
        0 \rightarrow (M_{i+1})_0 \rightarrow (M_i)_0 \rightarrow (J_{i + 1})_0 \rightarrow 0.
    \end{equation*}
    Так как $(J_{i + 1})_0 \simeq A_0$, то из свойсва аддитивности $A_0$-размерности (Предложение 2.11 книги 
    \cite{A-M}) имеют место равенства
    \begin{equation*}
        \dim_{A_0} (M_i)_0 - \dim_{A_0}A_0 = \dim_{A_0} (M_i)_0 - 1 = \dim_{A_0}(M_{i+1})_0.
    \end{equation*}
    Таким образом, последовательность \eqref{subsets} действительно обрывается. В базовом случае будем
    иметь точную тройку вида
    \begin{equation*}
        0 \rightarrow I_1 \rightarrow M_n \rightarrow I_2 \rightarrow 0, 
    \end{equation*}
    в которой для $A$-модуля $M_n$ уже можем вычислить кручения $\hat A$-модуля $\hat M_n$.

    Далее можно действовать индуктивно, где для шага индукции имеем точную тройку 
    \begin{equation*}
        0 \rightarrow M_{i-1} \rightarrow M_i \rightarrow I_{n - i + 1} \rightarrow 0,
    \end{equation*}
    которая позволяет вычислить $\Tors{\hat A}{\hat M_i}$, используя $\Tors{\hat A}{\hat M_{i-1}}$ и
    $\Tors{\hat A}{\hat I_{n - i + 1}}$ по формулам, аналогичным \eqref{torsHatM}.