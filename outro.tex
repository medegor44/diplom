В ходе работы были достигнуты поставленные цели: изучены понятия коммутативного кольца, идеала и модуля. Были выполнены упражнения из
книги \cite{A-M}, в ходе решения которых были доказаны различные условия при которых элементы конкретных колец 
обладают определенными свойствами (например, нильпотнетность или обратимость). 
Были получены явные формулы для вычисления тензорных и периодических произведений в простейших случаях.
В ходе работы были получены явные выражения для подмодуля кручения в некоторых тензорных произведениях,
вычислены делители нуля алгебры $\bigoplus_{s \geq 0}{(I[t] + (t))^s / (t^{s + 1})}$.