В ходе работы были выполнены поставленные задачи: изучены понятия коммутативного кольца, идеала и модуля. Выполнены упражнения из
книги \cite{A-M}, в ходе решения которых были доказаны различные условия при которых элементы конкретных колец 
обладают определенными свойствами (например, нильпотентность или обратимость). 
Доказан ряд свойств топологии Зарисского на спектре коммутативного кольца. 
Получены явные формулы тензорных и периодических произведений в простейших случаях.
Также в ходе работы получены явные выражения для подмодуля кручения в некоторых тензорных произведениях,
вычислены делители нуля алгебры $\bigoplus_{s \geq 0}{(I[t] + (t))^s / (t^{s + 1})}$.