В данной работе выполнен ряд задач, связанных с теорией модулей над коммутативным кольцом.
Были поставлены как исключительно учебные, так и задачи, происходящие из науных наработок руководителя.
\begin{enumerate}
    \item Изучить основные понятия и теоремы, связанные с коммутативными кольцами и идеалами.
    \item Выполнить ряд упраженений из книги \cite{A-M}.
    \item Изучить основные понятия и теоремы, связанные с модулями над коммутативным кольцом.
    \item Представить явные формулы для вычисления тензорных и периодических произведений в некоторых 
          простейших случаях.
    \item Вычислить кручение в тензорных произведениях вида $\left(\bigoplus_{s \ge 0} I^s \right) \ox_A M$.
    \item Вычислить делители нуля алгебры $\bigoplus_{s \geq 0}{(I[t] + (t))^s / (t^{s + 1})}$.
\end{enumerate}
Работа состоит из трех глав.

В первой главе работы будут рассмотрены кольца, идеалы и операции над ними. В этой же части приведены решения некоторых упражнений из главы 1 книги \cite{A-M}. В процессе 
решения упражнений были изучены такие понятия как радикал идеала, частное идеалов, расширение и сужение идеалов, понятие простого спектра кольца.

Во второй главе работы рассматриваются модули над заданным кольцом и опреации над ними. 
Вводится классическое понятия тензорного произведения, рассматриваются его свойства и даются 
явные формулы для вычисления тензорных произведений в некоторых простейших случаях. 
Далее приводится известная конструкция периодических произведений с помощью свободных резольвент и 
основанное на ней явное вычисление $\Tor_1^\ZZ(A, B)$, где $A$ и $B$ --- конечно порожденные абелевы группы.

Третья глава работы посвящена вычислению кручения в тензорных произведениях вида
$\left(\bigoplus_{s \ge 0} I^s \right) \ox_A M$, где $I \subset A$ --- идеал в кольце $A$, $M$ ---
$A$-модуль, $\bigoplus_{s \geq 0}{(I[t] + (t))^s / (t^{s + 1})} \ox_A J$, где $J \subset A$ --- идеал,
вычислению делителей нуля алгебры $\bigoplus_{s \geq 0}{(I[t] + (t))^s / (t^{s + 1})}$ с покомпонентным
умножением и вычислению кручения в $\left(\bigoplus_{s \ge 0} I^s \right) \ox_A M$, как $\hat A$-модуля.
