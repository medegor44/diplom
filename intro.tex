В данной работе будут рассмотрены понятия кольца, идеала, модуля и операций над ними. 
Основной целью работы является изучение указанных выше понятий. 
В ходе работы были решены некоторые задачи и упражнения из книги \cite{A-M}, а так же приведены 
явные формулы для вычисления тензорных и периодических произведений конечно порожденных абелевых групп. 

В первой части работы будут рассмотрены кольца, идеалы и операции над ними. В этой же части приведены решения некоторых упражнений из главы 1 книги \cite{A-M}. В процессе 
решения упражнений были изучены такие понятия как радикал идеала, частное идеалов, расширение и сужение идеалов, понятие простого спектра кольца.

Во второй части работы рассматриваются модули над заданным кольцом и опреации над ними. 
Вводится классическое понятия тензорного произведения, рассматриваются его свойства и даются 
явные формулы для вычисления тензорных произведений в некоторых простейших случаях. 
Далее приводится известная конструкция периодических произведений с помощью свободных резольвент и основанное на ней явное вычисление $\Tor_1^\ZZ(A, B)$, где $A$ и $B$ --- конечно порожденные абелевы группы.
